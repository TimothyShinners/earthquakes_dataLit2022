\documentclass{article}

% if you need to pass options to natbib, use, e.g.:
%     \PassOptionsToPackage{numbers, compress}{natbib}
% before loading neurips_2021

% ready for submission
\usepackage[preprint]{neurips_2021}

% to compile a preprint version, e.g., for submission to arXiv, add add the
% [preprint] option:
%     \usepackage[preprint]{neurips_2021}

% to compile a camera-ready version, add the [final] option, e.g.:
%     \usepackage[final]{neurips_2021}

% to avoid loading the natbib package, add option nonatbib:
%    \usepackage[nonatbib]{neurips_2021}

\usepackage[utf8]{inputenc} % allow utf-8 input
\usepackage[T1]{fontenc}    % use 8-bit T1 fonts
\usepackage[colorlinks=true]{hyperref}       % hyperlinks
\usepackage{url}            % simple URL typesetting
\usepackage{booktabs}       % professional-quality tables
\usepackage{amsfonts}       % blackboard math symbols
\usepackage{nicefrac}       % compact symbols for 1/2, etc.
\usepackage{microtype}      % microtypography
\usepackage{xcolor}         % colors

\title{A Small Attempt to Forecast Significant Earthquakes}

% The \author macro works with any number of authors. There are two commands
% used to separate the names and addresses of multiple authors: \And and \AND.
%
% Using \And between authors leaves it to LaTeX to determine where to break the
% lines. Using \AND forces a line break at that point. So, if LaTeX puts 3 of 4
% authors names on the first line, and the last on the second line, try using
% \AND instead of \And before the third author name.

\author{%
  Timothy Shinners\\
  Matrikelnummer 6116598\\
  \texttt{timothy.shinners@student.uni-tuebingen.de} \\
}

\begin{document}

\maketitle

\begin{abstract}
  For my project, I intend to use the Global Significant Earthquakes dataset from https://www.kaggle.com/mohitkr05/global-significant-earthquake-database-from-2150bc 
  
  It contains entries for over 6000 earthquakes and includes their date, magnitude, latitude and longitude. My current idea is to split the earthquakes into regions (I am not sure how I will go about this, perhaps I will use a clustering algorithm, perhaps I will attempt to group them by different fault lines). After this, I would like to assume that the occurrence of earthquakes follow a Poisson process, with different regions having different rates. Then, for each region x, I will calculate the average waiting time for an earthquake, after an earthquake occurs in region y, and I will see if this waiting time differs significantly from the overall average waiting time in region x. I am hoping to uncover some relationship where I can say "an earthquake in region x occurs, on average, once per year. However, if an earthquake occurs in region y, we expect the next earthquake in region x to occur within one month"
\end{abstract}



\end{document}
